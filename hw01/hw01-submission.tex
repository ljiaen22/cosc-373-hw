\documentclass{amsart}

\usepackage[vlined,linesnumbered]{algorithm2e}
\usepackage{enumitem}

\title{COSC-373: Homework 1}
\author{Lee Jiaen}
\author{Hyery Yoo}
\author{Alexander Lee}

\theoremstyle{definition}
\newtheorem{question}{Question}

\newcommand{\dist}{\text{dist}}

\begin{document}
\maketitle

\begin{question}
  \begin{enumerate}[label={(\alph*)}]
    \item Suppose the 3 students are $\{A, B, C\}$ and the 3 internships are
      $\{a, b, c\}$. The preferences for the students are:
      \begin{itemize}
        \item $A$: $a$, $c$, $b$
        \item $B$: $a$, $c$, $b$
        \item $C$: $c$, $a$, $b$
      \end{itemize}
      The preferences for the internships are:
      \begin{itemize}
        \item $a$: $C$, $A$, $B$
        \item $b$: $B$, $C$, $A$
        \item $c$: $A$, $B$, $C$
      \end{itemize}
      The 5 rounds proceed as follows:
      \begin{enumerate}[label={(\arabic*)}]
        \item $A$ applies to $a$, $B$ applies to $a$, and $C$ applies to $c$.
          $a$ defers $A$, $a$ rejects $B$, and $c$ defers $C$.
        \item $B$ applies to $c$. $c$ defers $B$ and $c$ rejects $C$.
        \item $C$ applies to $a$. $a$ defers $C$ and $a$ rejects $A$.
        \item $A$ applies to $c$. $c$ defers $A$ and $c$ rejects $B$.
        \item $B$ applies to $b$. $b$ defers $B$.
      \end{enumerate}
      The stable matchings found by the algorithm are thus $\{(A, c), (B, b),
      (C, a)\}$.
  \end{enumerate}
\end{question}

\begin{question}
  \begin{enumerate}[label={(\alph*)}]
    \item The stable matching found by the Gale-Shapley algorithm for such
      instances are $\{(s_1, u_1), (s_2, u_2), \ldots, (s_n, u_n)\}$. $n$ rounds
      are required until the algorithm terminates.
    \item Since running the Gale-Shapley algorithm with students applying to
      internships gives the same stable matching as running the Gale-Shapley
      algorithm with internships applying to students, the stable matching is
      unique.
  \end{enumerate}
\end{question}

\begin{question}
  \begin{enumerate}[label={(\alph*)}]
    \item The centralized greedy algorithm executes as follows.
      \begin{algorithm}[h]
        \DontPrintSemicolon%
        \SetKwFunction{neighbors}{neighbors}
        $M \gets \emptyset$\;
        $V_M \gets \emptyset$ \tcp{set of vertices that are matched in $M$}
        \ForEach{$v \in V$}{
          \ForEach{$u \in \neighbors{v}$}{
            \If{$v, u \notin V_M$}{
              $M \gets M \cup \{(v, u)\}$\;
              $V_M \gets V_M \cup \{v, u\}$\;
            }
          }
        }
        \Return{$M$}
      \end{algorithm}
    \item Since we are iterating over the neighbors for each vertex, we are thus
      iterating over the set of edges in $G$. Assuming $M$ and $V_M$ are hash
      sets, where lookup and add operations take $O(1)$ time, the runtime of our
      procedure is therefore $O(m)$.
    \item Let $G = (V, E)$ where $V = \{a, b, c, d\}$ and $E = \{(a,b), (b,c),
      (c,d)\}$. Two maximal matchings are $M_1 = \{(a,b), (c,d)\}$ and $M_2 =
      \{(b,c)\}$. Observe that $|M_1| = 2 |M_2|$.
  \end{enumerate}
\end{question}

\begin{question}
  Suppose for the sake of contradiction that $\dist(u, w) > \dist(u, v) +
  \dist(v, w)$. That is, the length of the shortest path from $u$ to $v$ plus
  the length of the shortest path from $v$ to $w$ is less than the length of the
  shortest path from $u$ to $w$. The assumption thus implies that there exists a
  path from $u$ to $w$ that is shorter than the shortest path from $u$ to $w$.
  Clearly, this is a contradiction. Therefore, $\dist$ satisfies the triangle
  inequality.
\end{question}

\end{document}
